\begin{abstracteng}
\textit{
This research presents a navigation application specifically designed for individuals with visual impairments, utilizing Bluetooth Low Energy (BLE) technology, the VOSK API for speech recognition, and the Kalman filter for enhanced position tracking. The proposed application aims to provide a reliable and intuitive navigation system for individuals with visual impairments to navigate indoor and outdoor environments with confidence and independence.The research leverages BLE technology to establish a connection between the user's mobile device and strategically placed beacons in the environment of the Faculty of Mathematics and Natural Sciences, Syiah Kuala University. BLE transmits location-specific information to the user's device, enabling precise indoor positioning and accurate navigation. By utilizing BLE technology, the application ensures a strong and reliable connection between the user and the surrounding environment.To enhance the user experience and provide a natural interaction, the VOSK API is integrated into the application. The VOSK API enables real-time speech recognition, allowing users to issue voice commands and receive audio feedback about their current location, nearby points of interest, and navigation instructions. This voice-based interaction eliminates the need for manual input and enables users to focus on their surroundings, improving safety and efficiency.The Kalman filter is utilized to improve the accuracy of the position tracking system. The filter combines measurements from various sensors such as accelerometer, gyroscope, and magnetometer to estimate the user's position with low error and good stability. By incorporating the Kalman filter, the application minimizes positioning errors and provides a smoother navigation experience for individuals with visual impairments.There are three main tests conducted in this research, including accuracy testing of navigation routes using Kalman Filter and Speech Command Recognition, functional testing of the application, and usability testing of the application. Based on the navigation route testing using the Kalman filter and Mean Squared Error (MSE), the selection of routes is influenced by BLE signal strength, user location, obstacles, and route suggestions. The lower the MSE, the more accurate the provided routes. Functional testing of the application using Black Box Testing yielded positive results, indicating that the developed application functions properly. The usability testing, conducted using the Usability Matrix for User Experience (UMUX), obtained a score of 78.33\%, indicating that the application is acceptable to users.}

\bigskip
\noindent
\textbf{\emph{Keywords :}} \textit{Bluetooth Low Energy}, \textit{Indoor Positioning System}, \textit{Fingerprinting}, \textit{VOSK API}, \textit{Kalman Filter}, \textit{Black Box}, \textit{UMUX}.
\end{abstracteng}