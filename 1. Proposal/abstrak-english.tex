\begin{abstracteng}
\noindent \textit{The proposed navigation application is designed specifically for individuals with visual impairments, utilizing Bluetooth Low Energy (BLE) technology, VOSK API for speech recognition, and Kalman filter for improved position tracking. The aim of this application is to provide a reliable and intuitive navigation system for individuals with visual impairments to navigate indoor and outdoor environments confidently and independently. This research leverages BLE technology to connect the user's mobile device with strategically placed beacons in the environment of the FMIPA Building, Syiah Kuala University. BLE provides specific location information to the user's device, enabling precise indoor positioning and accurate navigation. The integration of the VOSK API allows real-time speech recognition, allowing users to give voice commands and receive audio feedback regarding their current location and navigation instructions. This voice-based interaction eliminates the need for manual input and enhances safety and efficiency. The Kalman filter is utilized to improve the accuracy of the position tracking system. The research includes three main tests: accuracy testing of navigation routes using the Kalman filter and speech command recognition, functional testing of the application, and usability testing of the application. Based on the navigation route testing using the Kalman filter's Mean Squared Error (MSE), the route selection is influenced by BLE signal strength, user location, obstacles, and provided route suggestions. A lower MSE indicates more accurate route guidance. Functional testing of the application using Black Box Testing confirms that the developed application performs well. The usability testing results using the Usability Matrix for User Experience (UMUX) obtained a score of 78.33\%, indicating the application's acceptance by users.}






\bigskip
\noindent
\textbf{\emph{Keywords :}} \textit{Bluetooth Low Energy}, \textit{Indoor Positioning System}, \textit{Fingerprinting}, \textit{VOSK API}, \textit{Kalman Filter}, \textit{Black Box}, \textit{UMUX}.
\end{abstracteng}