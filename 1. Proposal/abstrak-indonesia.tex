\begin{abstractind}
Aplikasi navigasi yang dikhususkan untuk individu dengan gangguan penglihatan, menggunakan teknologi Bluetooth Low Energy (BLE), API VOSK untuk pengenalan ucapan, dan filter Kalman untuk meningkatkan pelacakan posisi. Aplikasi yang diusulkan bertujuan untuk memberikan sistem navigasi yang dapat digunakan untuk individu dengan gangguan penglihatan bergerak di lingkungan dalam dan luar dengan mudah dan mandiri.

Penelitian ini memanfaatkan teknologi BLE untuk menghubungkan perangkat seluler pengguna dengan beacon yang ditempatkan strategis di sekitar lingkungan Gedung FMIPA Universitas Syiah Kuala. BLE mengirimkan informasi yang spesifik terkait lokasi kepada perangkat pengguna, memungkinkan penentuan posisi dalam ruangan dengan presisi serta navigasi yang akurat. API VOSK memungkinkan pengenalan ucapan secara real-time, memungkinkan pengguna memberikan perintah suara dan menerima umpan balik audio mengenai lokasi saat ini dan petunjuk navigasi.Kalman filter digunakan untuk meningkatkan akurasi sistem pelacakan posisi.

Terdapat 3 pengujian utama pada penelitian ini meliputi pengujian akurasi rute navigasi menggunakan Kalman Filter dan Speech Command Recognition, pengujian fungsionalitas aplikasi dan pengujian usabilitas aplikasi. Berdasarkan pengujian rute navigasi menggunakan Kalman filter MSE (Mean Squared Error) pemilihan rute dipengaruhi oleh kekuatan sinyal BLE, lokasi pengguna, tembok, dan saran rute yang diberikan. Semakin rendah MSE maka semakin akurat rute yang diberikan. Pengujian fungsionalitas aplikasi dilakukan dengan menggunakan Black Box Testing mendapatkan hasil bahwa aplikasi yang dibangun berhasil berjalan dengan baik. Hasil yang didapatkan dari pengujian usabilitas menggunakan Usability Matrix for User Experience (UMUX) mendapatkan skor 78,33\% sehingga aplikasi dapat diterima oleh pengguna.





\bigskip
\noindent
\textbf{Kata kunci :} \textit{Bluetooth Low Energy}, \textit{Indoor Positioning System}, \textit{Fingerprinting}, \textit{VOSK API}, \textit{Kalman Filter}, \textit{Black Box}, \textit{UMUX}.
\end{abstractind}


