\preface % Note: \preface JANGAN DIHAPUS!


Segala puji dan syukur kepada Allah SWT yang telah melimpahkan rahmat serta hidayah-Nya kepada kita semua dan juga atas izin-Nya penulis dapat menyelesaikan penulisan proposal ini. Tak lupa Shalawat dan Salam penulis sanjung sajikan kepada Nabi Besar Muhammad SAW, karena beliau telah membawa kita semua dari alam jahiliah ke alam dengan ilmu pengetahuan.

Proposal yang berjudul \textbf{“Rancang Bangun Sistem Pengenalan Suara pada Aplikasi \textit{Route Guidance} untuk tunanetra berbasis \textit{Indoor Positioning}”} ini telah dapat diselesaikan atas bantuan banyak pihak. Oleh karena itu, melalui tulisan ini penulis ingin mengucapkan terima kasih dan penghargaan sebesar-besarnya kepada:

\begin{enumerate}
	\item{Bapak Kurnia Saputra, M.Sc., selaku Dosen Pembimbing I dan Bapak Alim Misbullah, S.Si., M.S., selaku Dosen Pembimbing II yang telah banyak memberikan bimbingan dan arahan kepada penulis, sehingga penulis dapat menyelesaikan Proposal ini.}
	\item {Bapak Dr. Muhammad Subianto, M.Si., selaku Ketua Jurusan Informatika.}
	\item {Bapak Muslim Amiren, S.Si., M.InfoTech., selaku Dosen Wali penulis.}
	\item {Seluruh Dosen di Jurusan Informatika Fakultas MIPA atas ilmu dan didikannya selama perkuliahan.}
	\item {Orang tua serta keluarga penulis yang telah membatu dan banyak memberikan dukungan secara spiritual, moral, dan material kepada penulis.}
	\item {Andika Pratama, Budi Gunawan, M. Zikri Aksnana, dan Yuda Aditya selaku teman yang telah banyak memberikan dukungan, masukan serta ilmu yang cukup besar dan bermanfaat dalam penulisan Proposal ini.}
	\item{Seluruh teman-teman seperjuangan Jurusan Informatika Unsyiah 2016 lainnya.}
\end{enumerate}


Penulis juga menyadari segala ketidaksempurnaan yang terdapat didalamnya baik dari segi materi, cara, ataupun bahasa yang disajikan. Seiring dengan ini penulis mengharapkan kritik dan saran dari pembaca yang sifatnya dapat berguna untuk kesempurnaan Proposal ini. Harapan penulis semoga tulisan ini dapat bermanfaat bagi banyak pihak dan untuk perkembangan ilmu pengetahuan.

\vspace{1cm}


\begin{tabular}{p{7.5cm}c}
	&Banda Aceh, November 2020\\
	&\\
	&\\
	&\\
	&\textbf{Penulis}
\end{tabular}